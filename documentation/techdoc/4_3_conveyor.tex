\documentclass[techdoc/techdoc.tex]{subfiles}

\begin{document}

\subsection{Conveyor}
Conveyor är en service som sköter all interaktion mellan EHR-modulen,
plattformsmodulen och gränssnittet. Den kallas Conveyor eftersom den kan liknas
lite till ett rullband. Hälsodata kommer från plattformar och lagras i
Conveyor. Därefter behandlas och modifieras datan medan den är i Conveyor.
Slutligen tas hälsodatan från Conveyor till EHR-modulen där den skickas vidare
till journalen.

Eftersom Conveyor endast överför och lagrar data består en av en väldigt liten
implementation. Conveyor är nästan helt agnostisk till de implementerade
plattformar. Conveyor behöver endast injicera varje plattform och därefter
kommunicera med det gränssnitt som är gemensamt för alla plattformar. Detta
gränssnitt beskrivs närmre i avsnitt \ref{sec:platform}. Conveyor är även helt
oberoende av vilka kategorier som är implementerade eftersom den endast lagrar
och överför datalistor som själva implementerar behandling och modifiering.

\subsubsection{Gränssnitt}
Metoden \texttt{getPlatforms} returnerar ett id i form av en sträng för varje
plattform som har injicerats av Conveyor och som går att använda. Det är det
här id:t som alltid används i gränssnittet för att specificera en plattform
till någon av metoderna hos Conveyor.

När en tillgänglig plattform väl är vald kan metoden \texttt{signIn} anropas
för att autentisera mot den givna plattformen. I fallet av Google Fit öppnas
här en extern pop-up som användaren kan logga in genom.

%TODO hur vet man när signIn är färdig?
När autentiseringen är genomförd går det att anropa
\texttt{getAvailableCategories} för att få reda på vilka kategorier som finns
tillgängliga hos den inloggade användaren på den valda plattformen. Även här
ges ett id i form av en sträng för varje tillgänglig kategori som används för
att specificera kategorin.

För att därefter hämta hälsodatan används funktionen \texttt{fetchData} som tar
emot plattformens id, \emph{ett} id för en kategori samt ett tidsintervall.
Metoden hämtar då en \texttt{Observable} från plattformen i fråga. Den binder
därefter den i en ny \texttt{Observable} som inte innehåller något och
returnerar den. När den sedan exekveras utförs hämtningen av hälsodatan och
lagras därefter i Conveyor. Klienten kan därmed få reda på när hämtningen är
klar, men datan sparas hos Conveyor istället för hos klienten. Klienten kan
sedan få en referens till datan genom att anropa \texttt{getDataList} med
kategorins id som inparameter.

Därefter kan klienten modifera datan fritt. Modifieringen av data sker ej via
Conveyor, utan görs direkt utifrån datalistan som Conveyor returnerar en
referens till när \texttt{getDataList} anropas.

När klienten väl är färdig med hämtning och modifiering kan den använda
\texttt{authenticateBasic} och \texttt{sendData} för att skicka vidare datan
till journalen.

\end{document}
