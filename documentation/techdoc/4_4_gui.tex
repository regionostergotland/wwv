\documentclass[techdoc/techdock.tex]{subfiles}

% Sidor:
%     home
%     help
%     info
%
% Statiska:
%     footer
%     toolbar
%
% Flöde:
%     progress-bar
%     sources (source-select)
%  -> category-picker (category-select)
%  -> sidebar (editor)
%         add-new-data-modal
%         bottom-sheet
%  -> inspection (review)
%
%     Gemensamma komponenter:
%         editor (data viewer)
%             health-list-items (data-table)
%             chart (data-chart)
%             add-data-point (datapoint-dialog)
%             math-dialog
%             remove-dialog

% beskriva implementation och struktur

\begin{document}

\subsection{GUI: grafiskt gränssnitt}
Alla grafiska element i applikationen består av Angular-komponenter. Hela sidan
som visas är applikationens AppComponent. Den innehåller statiska komponenter
som syns hela tiden såsom menypanelen. Den innehåller också en Routing-modul
som innehåller olika komponenter beroende på vilken URL som användaren befinner
sig på. Routingmodulen kan innehålla komponenter för huvudsidor eller vyer i
inrapporteringsflödet.

Det finns även en modul DataViewer som innehåller komponenter som används i
flera vyer. DataViewer ansvarar för att visualisera och interagera med
användarens data och används i både redigeringsvyn och inspektionsvyn.

Det grafiska gränssnittet använder sig till stor del av biblioteket Angular
Material för att få färdigbyggda komponenter. När en specifik komponent föregås
av prefixet \texttt{Mat} betyder det att det tillhör Material-biblioteket
(t.ex. \texttt{MatCard, MatDialog} o.s.v.).

\subsubsection{Statiska komponenter}
Applikationen har tre komponenter som alltid ligger i AppComponent oberoende av
nuvarande URL.

\paragraph{progress-bar}
ProgressBarComponent visar hur långt man kommit i inrapporteringsstegen och
vilka steg som återstår och syns då man startat en inrapporteringsprocess. Den
finns alltid i AppComponent men döljs om användaren inte befinner sig i
inrapporteringsflödet.

Den består av 3 knappar med motsvarande beskrivning. Knapparnas färg indikerar
vilket steg man är på och vilka steg som gjorts. Den använder nuvarande URL för
att bestämma vilket steg användaren är på.

\paragraph{ToolbarComponent}
visas alltid högst upp i applikationen, och innehåller en logogtyp samt länkar
till start, info och hjälpsidan. Består av en \texttt{MatToolbar} som
innehåller en logotyp och knappar som länkar till de olika sidorna.

\paragraph{FooterComponent}
är applikationens permanenta sidfot och innehåller för tillfället information
om licenser för ikoner.

\subsubsection{Komponenter för huvudsidor}
Applikationen har tre huvudsidor som ej är del av inrapporteringsflödet.

\paragraph{HomePageComponent}
är applikationens startsida. Härifrån kan man navigera till plattformsväljaren
(se \textit{sources}) för att välja hälsoplattform eller välja att gå till en
hjälpsida som förklarar hur hemsidan fungerar (se \textit{help}).

\paragraph{HelpPageComponent}
innehåller helt enkelt en kort användarguide som beskriver de olika stegen
en användare behöver göra för att skicka hälsodata.

\paragraph{InfoPageComponent}
är tänkt att visa information om applikationen, till exempel info om GDPR och
eventuellt kontaktuppgifter.

\subsubsection{Komponenter för inrapporteringsflöde}
Applikationen har fyra olika vyer och därmed fyra olika komponenter för
inrapporteringsflödet.

\paragraph{platform-selection}
visar vilka hälsoplattformar som finns tillgängliga så att användare kan välja
vilken plattform (t.ex. Google Fit) de vill hämta data ifrån.

För varje tillgänglig plattform visas ett \texttt{MatCard} med namnet på
plattformen samt dess logotyp. När man klickar på ett sådant \texttt{MatCard}
kallas typescriptfunktionen \textit{selectPlatform}, som då använder Conveyor
för att sköta inloggning till den valda plattformen.

Använder Conveyor för att hämta de hälsoplattformar som är tillgängliga samt
för att sköta inloggning då användare valt en plattform att hämta data.

\paragraph{category-selection} används för att välja de kategorier som ska
hämtas från en källa. Den använder sig av \texttt{Conveyor} för att hämta de
kategorier som finns tillgängliga att välja mellan, och sparar de id:n som är
valda i de checkboxes som ritas ut. Den väljer automatiskt en månad bakåt i
tiden som det tidsintervall för vilket data ska hämtas.

Komponenten består av två \texttt{MatCard}, en för att visa alla
valbara kategorier, och en för att välja datum. Kategorierna ritas ut m.h.a.
en for-loop som går igenom de kategorier som har hämtats från \texttt{Conveyor}.

\paragraph{editor-view}
hanterar vyn som tillåter användaren att redigera och komplettera sin
hälsodata. Den låter användaren redigera och se sin data genom att inkludera en
DataContainerComponent från DataViewerModule för varje kategori som användaren
har hämtat data för.

Dess AddNewDataModalComponent innehåller knappar för att lägga till kategorier
eller hämta ny data. Dess BottomSheetCategoriesComponent används för att välja
vilken kategori att lägga till om användaren vill mata in egen data.

\paragraph{inspection-view} är den komponent som ger granskningsvyn, det sista
steget innan man skickar in hälsodata till RÖD. Här kan man se all den data man
är på väg att skicka in, men man kan ej göra några ändringar. Alla kategorier
som man har fyllt med data visas i paneler som kan expanderas för att visa en
lista med all den data som lagts in på den kategorin. Denna komponent
innehåller även funktionalitet för autentisering mot RÖD. Här är tanken att
autentiseringen ska ske via BankID, men i nuläget sker den helt enkelt med
vanligt användarnamn och lösenord.

Består av en \texttt{mat-expansion-panel} för varje kategori som innehåller
data. I varje panels \texttt{mat\-expansion-panel-header} visas både namnet på
kategorin och hur många datavärden den innehåller. När en användare klickar på
en panel så expanderas den och visar en lista med alla datavärden för den
aktuella kategorin. Listan med data utgörs helt enkelt av en
DataContainerComponent, men med \texttt{editable} satt till false så att man ej
kan manipulera datavärdena på något sätt.

Resten av komponenten består av inputs för användarnamn och lösenord samt en
knapp som kallar på autentiseringsfunktionen i Conveyor med det användarnamn
och lösenord som fyllts i, samt funktionen \textit{sendData} i Conveyor.

Använder sig av Conveyor för att kunna hämta kategorier, se om en viss kategori
innehåller data eller ej, autentisera mot RÖD samt skicka hälsodata.

\subsubsection{DataViewerModule}
DataViewerModule innehåller komponenter för att visa och redigera hälsodata.
DataViewerModule inkluderar komponententerna DataTableComponent,
DataChartComponent och DataContationerComponent.

\paragraph{data-table} är den lista som visar alla datapunkter från en vald
kategori. Listan kan bli satt \texttt{editable}, för att kunna ändra på
kommentarer och dylikt.  Listan har en \texttt{paginator} för att den inte ska
bli allt för stor, samt beskrivningar av allt den visar upp som hints.

Komponenten består av en \texttt{MatTable} med en \texttt{paginator}.
Komponenten använder en for-loop för att få fram alla kolumner. Komponenter
använder en for-loop för att sätta visningen av alla datavärden, om det är
text, datum, drop-down-lista m.m.. Dessa är även olika om värdena är
\texttt{editable} eller ej. För att bestämma visningstyp används if-satser.

Komponenten hämtar \texttt{CategorySpec} från \texttt{Conveyor} och sätter
dessa i \texttt{displayedColumns} som används för uppritning.

\paragraph{data-chart}
är ett diagram som ritar ut alla numeriska värden för varje filter i den givna
kategorin.

\paragraph{data-container}
består av flikar där varje flik består av en DataTableComponent för varje
filter som användaren valt. En av flikarna är också en DataChartComponent som
visualiserar värdena för alla filter.

Komponenten innehåller också knappar för att lägga till, ta bort eller filtrera
punkter.

\paragraph{data-point-dialog} används som en pop-up, den är en
\texttt{MatDialog} och öppnas när användare vill lägga till eller modifiera en
datapunkt i en kategori.

Eftersom \texttt{data-point-dialog} är en \texttt{MatDialog} används MatDialogs
header, content, och footer. I headern visas kategorins titel. I content finns
ett antal if-satser som kollar vilken sorts data som ska matas in och visar
rätt sorts input. I footern finns två knappar: en för att lägga till data, och
en för att avbryta inmatningen, vilket även stänger komponenten.

DataPointDialogCompenent använder sig av en \texttt{CategorySpec} för att veta
vilka sorters data som ska läggas in i varje kategori. Kategorinamnet injekteras
av den komponent som använder sig av \texttt{data-point\-dialog}, där
\texttt{data-point-dialog} sedan använder sig av det namnet för att hämta
\texttt{CategorySpec} från \texttt{Conveyor}. Det finns en
\texttt{ErrorStateMatcher} som används för att skicka error-meddelanden till de
input-fält som inte är ifyllda.

\subsubsection{Responsiv design}
För att implementera en responsiv design användes till viss del Angulars Flex-Layout
bibliotek. Detta används genom att sätta attribut i HTML taggarna. 
t.ex. kommer \texttt{<div fxHide fxShow.lt-sm>small screen</div>} gömmas som default, och visas
när enhetens skärm uppfyller lt-sm (vilket motsvarar mediaQuerien \texttt{screen and (max-width: 599px)}).
Mer info om olika storlekar finns här \url{https://github.com/angular/flex-layout/wiki/Responsive-API}.
Ett problem som uppstod med denna taktik var t.ex. i tabellen. På den mobila versionen 
behövde vissa kolumner tas bort, för att visas i en modal istället. Att använda fxHide gömmer 
endast elementet, vilket i och för sig är intuitivt men detta gjorde att kolumnerna som
skulle flyttas till modalen fortfarande dök upp, dock med gömda element. 
Istället slopades kolumntitlarna i TypeScript filen genom kolla ifall ovanstående villkor 
uppfylldes. I HTML-komponenten användes istället Angulars inbyggda villkorliga utritning
ngIf som använde samma taktik för att bestämma ifall ett element skulle ritas ut.  

Ett problem som kvarstår för den mobila vyn är att det på vissa enhter ej är uppenbart 
att huvudytan (mellan header och kategorimenyn längst ned) är scrollbar, och att det 
finns knappar under huvudtabllen. Det hittades ingen bra lösning på detta, förutom att möjligen
inkludera det i instruktionenerna som visas när användaren först laddar denna vy. 



\end{document}
